\documentclass{scrartcl}
     
\usepackage[hidelinks]{hyperref}
\usepackage[none]{hyphenat}

\title{COMP210-Research Journal}
\subtitle{VR Real World Applications and Human Interface Design}

\author{1506530}

\begin{document}
	
	
\maketitle
\section{Journal}

\abstract { VR is here, and it seems to be here to stay. This journal contains only a small number of papers that relate to VR, the aspect of which is on the applications for VR in real world situations as well as the human interface design that was approached by each paper.}
		\newline
		\newline
		\textbf{Title: Virtual Reality for Skin Exploration}\cite{Vazquez}
		\newline
		This paper provides an interesting concept into using human interaction with VR, it proposes that using a hand held want and several 3D TV's that a medical student can be taught about the skin. Its a good approach to learning as its more interactive than a book, and can be used for a wide range of other applications. Though the concept is solid it could have its drawbacks with expenses, this could be lessened with the use of a VR headset. 
		\newline
		\newline
		\newline
		\textbf{Title: Enhancing Stress Management Techniques Using Virtual Reality}\cite{Soyka}
		\newline
		 The concept here is to provide a more effective way to relax in out ever more stressful world. The group utilized a VR headset coupled with an tranquil underwater scene to be used alongside controlled breathing to calm the user. The feedback they received was highly positive and it seems that this could be common place for people when they wish to de-stress. A very practical concept when looking into Human interface design and interactions.
		\newline
		\newline
		\newline
		\textbf{Title: From Driving Simulation to Virtual Reality}\cite{Kemeny}
		\newline
		 A practical application for a VR interface is for a driving simulator which is compared to VR. Driving simulators have been around for many years and even with the advances in technology the method behind these simulators stays the same. Although the main point here is to maximize efficiency and immersion, by reducing latency and increasing quality. The uses these variables in an attempt to reduce simulation sickness which is an issue across most VR interfaces. This is useful in the aiding of all VR technologies as the two points of improvement, 'latency and quality' are always being pushed forward.
		\newline
		\newline
		\newline
		\textbf{Title: HandNavigator: Hands-on Interaction for Desktop Virtual Reality}\cite{Kry}
		\newline
		An interesting paper, it proposes the use of hands on manipulation device that is placed on a surface os used by the users hand. It states that gloves and be tricky to use so they designed this device with various hand sizes in mind without worrying about which hand to use. This devise is then coupled with a VR hand that you can control to completed various tasks, though the unique property that this device has is the haptic feedback it gives, which can increase immersion and be more comfortable for the user. This ties into the human interfaces and interactions development nicely but providing a physical artifact to aid in the use of VR.
		\newline
		\newline
		\newline
		\textbf{Title: Virtual Reality Applications in Forensic Psychiatry}\cite{Benbouriche}
		\newline
		This project is a good cause, here they are attempting to use a VR interface to discover the cause and rehabilitate people with mental issues. As stated though its dificult for the section to receive attention and work form the community surrounding VR. This is both due to the political and funding issues coming from this kind of project, this is unfortunate as working towards this kind of medical assistance would benefit humanity as a whole and keep society safe from themselves. Though the time required to look into this method of VR interface would require too much time I would be very interested in conducting further research into this.
		\newline
		\newline
		\newline
		\textbf{Title: Virtual Reality Tools for the West Digital Conservatory of Archaeological Heritage}\cite{Barreau}
		\newline
		This paper shows another use for VR, the study of archeology. The VR headset is used to aid the researchers in their effort to understand our history, a 3D model of digsites and/or temples are created to allow the researchers to study the ruins without worrying about damaging the environment around them. The key here is to allow the archaeologists greater access to both VR and in turn the locations they wish to study. This could be implemented for underwater study also. Doing this would only require reference photos to create a 3D environment for them to explore, without the extra costs of sending people to these locations.

											
	\bibliographystyle{ieeetr}
	\bibliography{COMP210-Journal}
											
\end{document}