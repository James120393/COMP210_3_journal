\documentclass{scrartcl}

\usepackage[hidelinks]{hyperref}
\usepackage[none]{hyphenat}

\title{Essay Proposal}
\subtitle{COMP210 - Research Journal}

\author{1506530}

\begin{document}
	
	
	\maketitle
	\section*{Topic}
	
	
	\section*{Paper 1}
	\begin{description}
		\item[Title:] Virtual Reality for Skin Exploration
		\item[Citation:] \cite{Vazquez}
		\item[Abstract:] ``
		
		Background: We have developed and set up the SkinExplorer™ platform, a new tool to exploit and rebuild serial confocal images into 3D numerical models [1, 2]. The acquisitions using confocal microscopy allow visualizing cutaneous components as elastic fibers, melanocytes and keratinocytes etc... These diversified sources of data participate to create numerical 3D volume models with high quality of visualization.
		
		Objective: To create a Virtual Reality (VR) experience, to communicate and change the perception of skin structures by virtualization mode.
		
		Methods: The use of ART TRACKPACK system and ART SMARTTRACK device allow us to valorize new sensory images for the volumetric rendering of the 3D skin models.
		
		Results: We increase the perception and the understanding of skin components organization.
		
		Conclusion: The SkinExplorer™ platform seems to be a promising system for exploring the skin.
		''
		\item[Web link:] \url{http://doi.acm.org/10.1145/2466816.2466822}
		\item[Comments:] This paper provides an interesting concept into using human interaction with VR, it proposes that using a hand held want and several 3D TV's that a medical student can be taught about the skin. Its a good approach to learning as its more interactive than a book, and can be used for a wide range of other applications. Though the concept is solid it could have its drawbacks with expenses, this could be lessened with the use of a VR headset. 
	\end{description}
	
	\section*{Paper 2}
	\begin{description}
		\item[Title:] Enhancing Stress Management Techniques Using Virtual Reality
		\item[Citation:] \cite{Soyka}
		\item[Abstract:] ``Chronic stress is one of the major problems in our current fast paced society. The body reacts to environmental stress with physiological changes (e.g. accelerated heart rate), increasing the activity of the sympathetic nervous system. Normally the parasympathetic nervous system should bring us back to a more balanced state after the stressful event is over. However, nowadays we are often under constant pressure, with a multitude of stressful events per day, which can result in us constantly being out of balance. This highlights the importance of effective stress management techniques that are readily accessible to a wide audience. In this paper we present an exploratory study investigating the potential use of immersive virtual reality for relaxation with the purpose of guiding further design decisions, especially about the visual content as well as the interactivity of virtual content. Specifically, we developed an underwater world for head-mounted display virtual reality. We performed an experiment to evaluate the effectiveness of the underwater world environment for relaxation, as well as to evaluate if the underwater world in combination with breathing techniques for relaxation was preferred to standard breathing techniques for stress management. The underwater world was rated as more fun and more likely to be used at home than a traditional breathing technique, while providing a similar degree of relaxation.''
		\item[Web link:] \url{http://doi.acm.org.ezproxy.falmouth.ac.uk/10.1145/2931002.2931017}
		\item[Comments:] The concept here is to provide a more effective way to relax in out ever more stressful world. The group utilized a VR headset coupled with an tranquil underwater scene to be used alongside controlled breathing to calm the user. The feedback they received was highly positive and it seems that this could be common place for people when they wish to de-stress. A very practical concept when looking into Human interface design and interactions.
	\end{description}
	
	\section*{Paper 3}
	\begin{description}
		\item[Title:] From Driving Simulation to Virtual Reality
		\item[Citation:] \cite{Kemeny}
		\item[Abstract:] ``
		
		Driving simulation from the very beginning of the advent of VR technology uses the very same technology for visualization and similar technology for head movement tracking and high end 3D vision. They also share the same or similar difficulties in rendering movements of the observer in the virtual environments. The visual-vestibular conflict, due to the discrepancies perceived by the human visual and vestibular systems, induce the so-called simulation sickness, when driving or displacing using a control device (ex. Joystick). Another cause for simulation sickness is the transport delay, the delay between the action and the corresponding rendering cues.
		
		Another similarity between driving simulation and VR is need for correct scale 1:1 perception. Correct perception of speed and acceleration in driving simulation is crucial for automotive experiments for Advances Driver Aid System (ADAS) as vehicle behavior has to be simulated correctly and anywhere where the correct mental workload is an issue as real immersion and driver attention is depending on it. Correct perception of distances and object size is crucial using HMDs or CAVEs, especially as their use is frequently involving digital mockup validation for design, architecture or interior and exterior lighting.
		
		Today, the advents of high resolution 4K digital display technology allows near eye resolution stereoscopic 3D walls and integrate them in high performance CAVEs. High performance CAVEs now can be used for vehicle ergonomics, styling, interior lighting and perceived quality. The first CAVE in France, built in 2001 at Arts et Metiers ParisTech, is a 4 sided CAVE with a modifiable geometry with now traditional display technology. The latest one is Renault's 70M 3D pixel 5 sides CAVE with 4K x 4K walls and floor and with a cluster of 20 PCs. Another equipment recently designed at Renault is the motion based CARDS driving simulator with CAVE like 4 sides display system providing full 3D immersion for the driver.
		
		The separation between driving simulation and digital mockup design review is now fading though different uses will require different simulation configurations.
		
		New application domains, such as automotive AR design, will bring combined features of VR and driving simulation technics, including CAVE like display system equipped driving simulators.
		''
		\item[Web link:] \url{http://doi.acm.org/10.1145/2617841.2620721}
		\item[Comments:] Here we have a practical application for a VR interface where a driving simulator is compared to VR. Driving simulators have been around for many years and even with the advances in technology the method behind these simulators stays the same. Although the main point here is to maximize efficiency and immersion, by reducing latency and increasing quality. The uses these variables in an attempt to reduce simulation sickness which is an issue across most VR interfaces. This is useful in the aiding of all VR technologies as the two points of improvement, 'latency and quality' are always being pushed forward.
	\end{description}
	
	\section*{Paper 4}
	\begin{description}
		\item[Title:] HandNavigator: Hands-on Interaction for Desktop Virtual Reality
		\item[Citation:] \cite{Kry}
		\item[Abstract:] ``This paper presents a novel interaction system, aimed at hands-on manipulation of digital models through natural hand gestures. Our system is composed of a new physical interaction device coupled with a simulated compliant virtual hand model. The physical interface consists of a SpaceNavigator, augmented with pressure sensors to detect directional forces applied by the user's fingertips. This information controls the position, orientation, and posture of the virtual hand in the same way that the SpaceNavigator uses measured forces to animate a virtual frame. In this manner, user control does not involve fatigue due to reaching gestures or holding a desired hand shape. During contact, the user has a realistic visual feedback in the form of plausible interactions between the virtual hand and its environment. Our device is well suited to any situation where hand gesture, contact, or manipulation tasks need to be performed in virtual. We demonstrate the device in several simple virtual worlds and evaluate it through a series of user studies.''
		\item[Web link:] \url{http://doi.acm.org/10.1145/1450579.1450591}
		\item[Comments:]This is an interesting paper that proposes the use of hands on manipulation device that is placed on a surface os used by the users hand. It states that gloves and be tricky to use so they designed this device with various hand sizes in mind without worrying about which hand to use. This devise is then coupled with a VR hand that you can control to completed various tasks, though the unique property that this device has is the haptic feedback it gives, which can increase immersion and be more comfortable for the user. This ties into the human interfaces and interactions development nicely but providing a physical artifact to aid in the use of VR.
	\end{description}
	
	\section*{Paper 5}
	\begin{description}
		\item[Title:] Virtual Reality Applications in Forensic Psychiatry
		\item[Citation:] \cite{Benbouriche}
		\item[Abstract:] ``Violent offending behaviours remain an important issue in particular when associated with mental illness. To prevent recidivism and protect society, investments are required to develop new tools that would provide decision makers with a better understanding of violent behaviours and ultimately improve treatment options for violent offenders. Recently, Virtual Reality (VR) is gaining recognition as promising tool in forensic psychiatry. Amongst other things, VR allows a renewal from both methodological and theoretical points of view. The aim of this paper is to introduce VR applications in the context of forensic psychiatry. After a brief introduction to the purpose of forensic psychiatry, examples will be given in order to illustrate how VR can help address some of the field's current issues.''
		\item[Web link:] \url{http://doi.acm.org/10.1145/2617841.2620692}
		\item[Comments:] This project is a good cause, here they are attempting to use a VR interface to discover the cause and rehabilitate people with mental issues. As stated though its dificult for the section to receive attention and work form the community surrounding VR. This is both due to the political and funding issues coming from this kind of project, this is unfortunate as working towards this kind of medical assistance would benefit humanity as a whole and keep society safe from themselves. Though the time required to look into this method of VR interface would require too much time I would be very interested in conducting further research into this.
	\end{description}
	
	\section*{Paper 6}
	\begin{description}
		\item[Title:] Virtual Reality Tools for the West Digital Conservatory of Archaeological Heritage
		\item[Citation:] \cite{Barreau}
		\item[Abstract:] ``In the continuation of the 3D data production work made by the WDCAH, the use of virtual reality tools allows archaeologists to carry out analysis and understanding research about their sites. In this paper, we focus on the virtual reality services proposed to archaeologists in the WDCAH, through the example of two archaeological sites, the Temple de Mars in Corseul and the Cairn of Carn Island.''
		\item[Web link:] \url{http://doi.acm.org/10.1145/2617841.2617845}
		\item[Comments:]This paper show another use for VR, the study of archeology. The VR headset is used to aid the researchers in their effort to understand our history, a 3D model of digsites and/or temples are created to allow the researchers to study the ruins without worrying about damaging the environment around them. The key here is to allow the archaeologists greater access to both VR and in turn the locations they wish to study. This could be implemented for underwater study also. Doing this would only require reference photos to create a 3D environment for them to explore, without the extra costs of sending people to these locations.
	\end{description}
	
	\section*{Paper 7}
	\begin{description}
		\item[Title:] Teaching Augmented Reality in Practice: Tools, Workshops and Students' Projects
		\item[Citation:] \cite{Wichrowski}
		\item[Abstract:] ``Increasing popularity of Augmented Reality (AR) and easy access to new software for creating projects using this technology leads to considerations how to integrate it with the students' assignments/projects and encourages us to check potential of it in the academic world. The first part of this paper gives a brief description of the AR technology, requirements needed for using AR, and the preparation process. The second part presents various editors, applications, tools and developing environments for teaching AR, useful in IT and art areas. It focuses on projects prepared by the students of Computer Science and New Media Art in the Polish-Japanese Information Technology, and also on projects created during AR workshops. The main goal of this paper is to present ideas and suggestions concerning applying AR technology to students' works based on the experience of the author, and final results are also shown.''
		\item[Web link:] \url{http://doi.acm.org/10.1145/2500342.2500362}
		\item[Comments:]
	\end{description}
	
	\section*{Paper 8}
	\begin{description}
		\item[Title:] Haptic Retargeting: Dynamic Repurposing of Passive Haptics for Enhanced Virtual Reality Experiences
		\item[Citation:] \cite{Azmandian}
		\item[Abstract:] ``Manipulating a virtual object with appropriate passive haptic cues provides a satisfying sense of presence in virtual reality. However, scaling such experiences to support multiple virtual objects is a challenge as each one needs to be accompanied with a precisely-located haptic proxy object. We propose a solution that overcomes this limitation by hacking human perception. We have created a framework for repurposing passive haptics, called haptic retargeting, that leverages the dominance of vision when our senses conflict. With haptic retargeting, a single physical prop can provide passive haptics for multiple virtual objects. We introduce three approaches for dynamically aligning physical and virtual objects: world manipulation, body manipulation and a hybrid technique which combines both world and body manipulation. Our study results indicate that all our haptic retargeting techniques improve the sense of presence when compared to typical wand-based 3D control of virtual objects. Furthermore, our hybrid haptic retargeting achieved the highest satisfaction and presence scores while limiting the visible side-effects during interaction.''
		\item[Web link:] \url{http://doi.acm.org/10.1145/2858036.2858226}
		\item[Comments:]
	\end{description}
	
	\section*{Paper 9}
	\begin{description}
		\item[Title:] Design and Implementation of a Haptics-based Virtual Venepuncture Simulation and Training System
		\item[Citation:] \cite{Xia}
		\item[Abstract:] ``Venepuncture or venipuncture is drawing blood from vein for testing or blood transfusion purposes. It is one of the most routinely performed invasive procedures that medical students must learn. Haptic interaction in virtual reality environments may provide an advance tool for training these skills. We present a novel and low-cost approach for image-based virtual haptic venepuncture simulation. We use actual photos of patient arms to provide a quick implementation of different virtual arms with better visual immersion. The function-based model of virtual veins helps us to achieve fast collision detection, while the haptic model for the multi-layer soft tissue provides stable and realistic force feedback. The implemented system was validated by medical staff.''
		\item[Web link:] \url{http://doi.acm.org/10.1145/2407516.2407523}
		\item[Comments:]
	\end{description}
	
	\section*{Paper 10}
	\begin{description}
		\item[Title:] An Education Method for VR Content Creation Using Groupwork
		\item[Citation:] \cite{Miyata}
		\item[Abstract:] ``
		VR content creation is a comprehensive development, and it requires a variety of skills, not only sensing technology and computer graphics techniques, but also aesthetic design and storytelling, for completing the project. A groupwork-based project is a suitable approach for creating a VR application, because the group members can exert their full powers in their special fields by collaborating with each other.
		
		Students learn best when they are actively involved in the process, such as in group discussion and field work. These groupwork projects are also effective in improving their collaboration skills.
		
		This paper introduces an education method for creating virtual reality content by means of groupwork, and shows the advantages of this method.
		''
		\item[Web link:] \url{http://doi.acm.org/10.1145/1666611.1666621}
		\item[Comments:]
	\end{description}
	
	\section*{Paper 11}
	\begin{description}
		\item[Title:] Immersive Environment for Robotic Tele-operation
		\item[Citation:] \cite{Bugalia}
		\item[Abstract:] ``
		In various modern day situations, controlling a robot from a remote location is essential due to hazardous environmental conditions (for human operators) near the robot. Thus arises the need for an intuitive user interface for tele-operation, which must be efficient as well as easy to use. In this paper we present an innovative user interface and overall framework for robotic tele-operation and demonstrate its application to simple bin-picking and hole-packing tasks. We have adopted technologies from Virtual Reality (VR) systems for environment mapping and used modern interface devices to provide haptic feedback.
		
		The user interface of our framework renders a virtual replica of the remote site in which the virtual objects are animated based on the tracking information received from cameras and robot placed at the remote location. A haptic device is used by the human operator to control the remote robotic arm while simultaneously aided by the haptic feedback received from the robotic arm. A tele-operation system using our framework is developed in laboratory environment and the usability of our system is verified by a user survey.
		''
		\item[Web link:] \url{http://doi.acm.org/10.1145/2783449.2783498}
		\item[Comments:]
	\end{description}

	

											
	\bibliographystyle{apalike}
	\bibliography{COMP210-Journal}
											
\end{document}